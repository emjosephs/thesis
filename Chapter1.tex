\setlength{\parindent}{0ex}
\setlength{\parskip}{2ex}

\chapter{General introduction}

\say{No one supposes that all the individuals of the same species are cast in the very same mould. These individual differences are highly important for us, as they afford materials for natural selection to accumulate} 
\begin{flushright}
--- \citet{darwin1859}
\end{flushright}

\say{The individuals belonging to a species differ to a greater or less extent”} 
\begin{flushright}
--- \citet{haldane1932}
\end{flushright}

\say{Attempts to understand the causes and significance of organic diversity have been made ever since antiquity; the problem seems to possess an irresistible aesthetic appeal.} 
\begin{flushright}
--- \citet{dob1937}
\end{flushright}

\say{“There is a considerable amount of genic variation segregating in all of the populations studied and .. the real variation in these populations must be greater than we are able to demonstrate.”}
\begin{flushright}
--- \citet{Lewontin1966-kz}
\end{flushright}

As many great evolutionary biologists have noted, natural variation is ubiquitous. However, the relative importance of various evolutionary forces in maintaining this variation, especially at the population level, is still unknown. 

\section{Population genetic approaches to the maintenance of variation.}

The question of what forces maintain genetic variation has been investigated extensively in population genetics. Two competing hypotheses formed early on for explaining within-population variation (reviewed in \citet{lewontin1974}). The “classical” school argued that little variation segregates in populations, most new mutations are deleterious, and most observed variation is transient, on the way to either adaptive fixation or removal. In contrast, the “balance” school argued that large amounts of genetic variation are maintained within populations due to heterosis or other forms of balancing selection. 

As the first protein polymorphism  data emerged, it appeared that neither hypothesis could explain the data \citep{Lewontin1966-kz,Harris1966-lm}. Levels of heterozygosity in proteins were too high to be explained by balancing selection alone since at every polymorphic site, homozygotes will still make up at least 50\% of the population at Hardy-Weinberg equilibrium, leading to extreme variation in fitness between individuals and inbreeding depression \citep{lewontin1974}. However, high levels of sequence variation did not fit with the original classical theory\textsc{\char13}s prediction that there is low variation within populations. Kimura and Ohta reconciled high within-population variation with the classical school by proposing that most of the sequence variation observed in nature both within and between populations is either selectively neutral or effectively selectively neutral \citep{Kimura1968-cl,Mootoo_Kimura1971-dc,Kimura_undated-by,Ohta1973-qx}.

However, genome-wide sequence data has shown that selection does shape observed genetic variation and molecular divergence between species, both directly in coding and noncoding sequence, and indirectly through linked selection. McDonald-Kreitman based approaches have found that \textgreater 30\% of replacement fixations in \textit{Drosophila} and \textit{Escherichia coli} are due to positive selection \citep{Eyre-Walker2006-jg,Fay2002-au,Begun2007-gh}. Within population estimates of the distribution of fitness effects of segregating mutations show that weakly deleterious mutations contribute significantly to standing genetic variation in humans \citep{Williamson2005-ja,Eyre-Walker2006-tr}. 

There is also evidence that selection shapes genetic variation outside genic regions. \citet{Begun2007-gh} found constraint across intergenic, intronic, and untranslated regions of the \textit{D. simulans} genome and there is constraint in the noncoding sequences that flank genes of other species \citep{Eory2010-ja}. However, quantifying selection in noncoding regions is difficult because these regions do not break down into obvious selected and neutral sites. Methods that have used information about specific function have been able to detect selection on functional sites such as transcription factor binding sites \citep{Arbiza2013-te} and phylogenetically-conserved noncoding sequences \citep{Halligan2013}. 

There is mounting evidence that selection shapes diversity at neutral sites in linkage disequilibrium with directly selected sites. Linked neutral diversity will be removed by both fixations due to positive selection \citep{Smith1974} and the removal of polymorphisms by negative selection \citep{Charlesworth1993-xx}. The combined effects of selective sweeps and background selection are widespread, as demonstrated by positive correlations between recombination rate and polymorphism \citep{Cutter2003-yq,Nachman1998-vc,Kim2007-ju,Begun2007-gh,Tenaillon2001-ai}, negative correlations between divergence and polymorphism \citep{Andolfatto2007-ew,Hahn2008-yj,Macpherson2007-pl}, and reduced average neutral diversity around replacement fixed substitutions relative to silent fixed substitutions \citep{Sattath2011-ns} but see \citep{hernandez2011}. Recent estimates suggest that linked selection dramatically reduces genome-wide diversity in the \textit{Drosophila simulans} genome \citep{Elyashiv2014-ic}.

Most of the discussion above focussed on negative and positive selection, which both remove genetic variation from within populations, but selection can also directly increase genetic variation through balancing selection. However, the impact of balancing selection on sequence variation has generally been difficult to quantify. Balancing selection can result from a number of different processes, including heterosis, frequency-dependent selection, spatially-variable selection, and temporally-variable selection \citep{Hedrick2006-ft,Hedrick1976-qp} and the temporal scale at which balancing selection occurs will affect the types of genetic signatures it leaves behind \citep{Charlesworth2006-mw}. For example, long-term balancing selection appears to maintain genetic variation between humans and chimpanzees at a number of loci \citep{Leffler2013-wr} but the genetic signature of short-term balancing selection is difficult to disentangle from other demographic events \citep{Charlesworth2006-mw}.

Overall, it is clear that some genetic variation is neutral, some is deleterious, and some is beneficial. However, the relative importance of these processes for shaping genome-wide variation is still unclear.

\section{Quantitative genetic approaches to the maintenance of variation}
The types of selection that shape sequence variation may differ from the types of selection that shape phenotypic variation because much of the variation at sequence level may not translate into phenotype (this is a central tenet of the nearly neutral theory). Since selection generally acts on the phenotype, understanding the selective forces that maintain genetic variation for phenotype is crucial. The main hypotheses for the maintenance of quantitative genetic variation for phenotypic traits mirror those in population genetics: mutation-selection balance and balancing selection \citep{N_H_Barton1989-yu,Johnson2005-dl}.

An exemplar of the kinds of approaches that quantitative geneticists have used to address the maintenance of genetic variation comes from Kelly, who proposed that the contribution of deleterious alleles to genetic variation could be quantified by investigating the directional dominance of variation in a trait \citep{Kelly1999-re}. Applications of this approach have shown that intermediate-frequency alleles, not rare recessive alleles as expected under mutation-selection balance, can explain variation for flower size in \textit{Mimulus guttatus} and female fecundity in \textit{Drosophila} \citep{Kelly2001-rc,Charlesworth2007-rp}. However, this approach requires multiple generations of breeding in order to measure directional dominance and has so far only been applied to a few traits that may not be representative of all traits. 

\section{Linking genotype and phenotype}
Developing an understanding of the selective forces that maintain genetic variation for phenotype on a systematic level will require linking genetic variation to phenotypic variation. An enormous effort has been made over the last few decades to do exactly this, using techniques like linkage mapping, quantitative trait loci mapping, and genome-wide association studies (GWAS). I will refer to all of these techniques under the umbrella term “QTN program”. following \citet{Rockman2012-ks}.

Often the QTN program has not been motivated by evolutionary questions, but instead by an attempt to better understand the molecular basis of a trait and then use this information to develop treatments for a disease, find genetic regions that could be introgressed into crop plants, or other practical uses. However, evolutionary biologists have also made use of QTN techniques to describe the genetic architecture of traits \citep{lynch1998}. It seems especially appealing to use the sample of known QTNs (sometimes distressingly referred to as the “QTLome” \citep{Martinez_undated-wg}) to make general conclusions about the types of variation maintained within populations \citep{Barton2002-do}.

Unfortunately, inferring the selective forces acting on variation from QTNs is not straightforward \citep{Rockman2012-ks,Johnson2005-dl}. The act of finding QTNs biases us towards finding large-effect alleles but other lines of evidence suggest that small-effect QTNs are also important for genetic variation. For example, the inability of genome-wide association mapping studies to identify large-effect common alleles for many heritable human diseases suggests that either small-effect and/or very rare large-effect alleles make up genetic variation for these traits (reviewed in Rockman 2012). 

As emphasized by \citep{Lee2014-pi}, identifying QTNs within a population is still a useful approach to understanding how selection acts on traits. Specifically, QTNs maintained by mutation-selection balance are expected to be found at low allele frequencies while QTNs maintained by balancing selection are expected to be found at intermediate allele frequencies \citep{Barton2002-do}. However, efforts to infer selective forces acting on QTNs must be careful of the ascertainment biases that filter the sample of known QTNs. One successful approach has been a set of studies that estimate the proportion of variation in a trait explained by all genotyped loci to show that small-effect alleles do contribute to genetic variation, even though the specific loci involved are unknown \citep{Yang2010-iu,International_Schizophrenia_Consortium2009-ks}.

A potential strategy for generating a sample of QTNs with relatively small fitness effects is to investigate the loci that affect gene expression. While expression is often thought of as an intermediate phenotype, it is more useful in this case to think of it as a ‘model phenotype’ that integrates across phenotypic space, is straightforward to measure on a large scale, and is measurable without prior, potentially biasing information about how a certain expression level may vary in nature or influence fitness \citep{Rockman2006-yx}.

\section{The importance of standing variation for adaptation}

While understanding the maintenance of variation is fundamental for a full understanding of evolutionary biology, it is also relevant because the types of variation maintained within populations will control adaptation. The classical theory predicts that adaptation occurs through new mutations, leading to the signatures of “classic” selective sweeps and an adaptive walk that occurs in short mutation-limited bursts \citep{Smith1974,Allen_Orr2005-fs,Fisher1930-el}. However, the balance theory suggests that ecologically relevant variation waits within populations, ready to contribute to adaptation if conditions change. Adaptive mutations that rise to fixation from standing variation will leave a subtler sweep signal in linked neutral variation since more than one haplotype may fix in the population \citep{pennings2006_2,pennings2006_3,hermisson2005}. If adaptation occurs at many loci present in standing variation, the population can reach a new trait optimum without any fixations at all \citep{Pritchard2010-uh,Pritchard2010-uq}.

The origin of adaptive alleles will not only affect our ability to detect that adaptation using population genetic data, but will also affect the kinds of mutations that actually fix during adaptation. For example, adaptation from standing variation may involve more small-effect alleles than adaptation from new mutation \citep{Orr2001-fq}. In addition, local adaptation between divergent populations from standing variation may involve more instances of conditional neutrality, where one allele is selected in one population and neutral in another population, than trade-offs, where alleles are alternately favored and disfavored in populations, since the alleles involved in trade-offs are less likely to be maintained within populations than neutral alleles \citep{Anderson2011-hs}. 

\section{Research Objectives}

Throughout the course of my Ph.D., I have investigated the selective forces that maintain genetic variation within populations and that shape sequence variation at the population, within-species, and between-species level. Below I describe my study system and outline the chapters of my thesis.

\subsection{Study System}
I used the species \textit{Capsella grandiflora} throughout my thesis. \textit{C. grandiflora} has been a useful system for the investigation of selective forces for a number of reasons. First, \textit{C. grandiflora} has had a large, relatively stable effective population size making both positive and negative selection strong relative to drift \citep{Slotte2010-gw}. Second, \textit{C. grandiflora} has relatively low population structure \citep{St_onge2011-jz}, making it possible to do population genomics on a species-wide sample and minimizing the impact that migration from nearby populations will have on a focal population. Third, unlike most plant model systems, \textit{C. grandiflora} is obligately outcrossing, making it appropriate for answering questions about selective forces that act within populations. Finally, \textit{C. grandiflora} is a highly tractable genomic system. It has a small genome for a plant (219 MB), making it feasible to generate and map sequence data from a population sample and is closely related to \textit{Arabidopsis thaliana} ($\sim$10 million years ago), allowing me to use well-curated annotation information on gene function and structure.

\subsection{Chapter 2}

In this chapter I tested for the signature of recurrent selective sweeps in coding and conserved non-coding regions of the \textit{C. grandiflora} genome and described the effects of expression level on the strength of selection in genic regions. This work was part of a collaborative effort with Robert Williamson to describe the strength of positive and negative selection across the \textit{C. grandiflora} genome. 

Robert used the site frequency spectrum to estimate the distribution of fitness effects of new mutations and, ultimately, quantify the relative strength of selection across the \textit{C. grandiflora} genome. He found that both positive and negative selection appeared to be strong in coding and conserved noncoding regions. I detected reduced neutral diversity surrounding fixed replacement substitutions in \textit{C. grandiflora} relative to fixed synonymous substitutions, consistent with the action of recurrent selective sweeps. I also detected the signature of recurrent selective sweeps around conserved noncoding sequences.

In addition, I investigated the importance of expression level in shaping between-gene variation in selection. Previous work has shown that highly expressed genes diverge more slowly than lowly expressed genes. However, this could result from either stronger constraint on highly expressed genes or stronger positive selection on low expressed genes. I showed that stronger negative selection in highly expressed genes explains the observed correlation between expression and divergence.

This work was published in Plos Genetics in 2014: Williamson RJ*, Josephs EB*, Platts AE, Hazzouri KM, Haudry A, Blanchette M, et al. (2014) Evidence for Widespread Positive and Negative Selection in Coding and Conserved Noncoding Regions of \textit{Capsella grandiflora}. PLoS Genet 10(9): e1004622. doi:10.1371/journal.pgen.1004622 (*The two first authors contributed equally)

\subsection{Chapter 3}
	In this chapter I set out to map the genetic variants that affect expression in a population of \textit{C. grandiflora} and used the allele frequencies and effect sizes of these variants to show that negative selection acts on these variants, and ultimately, on expression variation. This project was done in collaboration with Young Wha Lee, who mapped the genomic data and called variants. A preprint is posted on Biorxiv: Josephs, EB, Lee, YW, Stinchcombe, HR, \& Wright, SI. Association mapping reveals the role of mutation-selection balance in the maintenance of genomic variation for gene expression  http://biorxiv.org/content/early/2015/09/21/015743

\subsection{Chapter 4}
	In this chapter I investigated the relationship between expression network structure and selection. I collaborated with Dan Schoen at McGill University, who used the expression data generated in Chapter 3 to construct coexpression networks. I then used these networks to test for the effects of network connectivity and module membership on selection. I found that connectivity increased both positive and negative selection, suggesting that pleiotropy, as measured through connectivity, does not constrain adaptation. 
